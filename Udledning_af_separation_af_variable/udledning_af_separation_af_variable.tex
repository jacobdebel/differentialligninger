% Intended LaTeX compiler: xelatex
\documentclass[a4paper, 10pt]{article}
\usepackage{graphicx}
\usepackage{grffile}
\usepackage{longtable}
\usepackage{wrapfig}
\usepackage{rotating}
\usepackage[normalem]{ulem}
\usepackage{amsmath}
\usepackage{textcomp}
\usepackage{amssymb}
\usepackage{capt-of}
\usepackage{hyperref}
\usepackage[danish]{babel}
\usepackage{mathtools}
\usepackage[left=1.0cm,right=1.0cm,top=0.0cm,bottom=1.5cm]{geometry}
\hypersetup{colorlinks, linkcolor=black, urlcolor=blue}
\setlength{\parindent}{0em}
\parskip 1.5ex
\author{Matematik A}
\date{Vibenshus Gymnasium}
\title{Bevis for separation af de variable til løsning af differentialligninger}
\hypersetup{
 pdfauthor={Matematik A},
 pdftitle={Bevis for separation af de variable til løsning af differentialligninger},
 pdfkeywords={},
 pdfsubject={},
 pdfcreator={Emacs 27.2 (Org mode 9.4.4)}, 
 pdflang={Danish}}
\begin{document}

\maketitle
Dette dokument er et bevis for og et regneeksempel på brugen af separation af variable i forbindelse med løsning af 1. ordens ordinære differentialligninger. Venstre side viser det generelle bevis, mens højre side viser et eksempel.

\fbox{
\begin{minipage}{0.48\textwidth}
Lad os antage, at vi har en 1. ordens ordinær differentialligning på formen:

$$\frac{d y}{dx} = g(x) \cdot h(y)\,.$$

Denne omarrangeres til

$$\frac{1}{h(y)} \cdot \frac{d y}{dx} = g(x)\,.$$

Nu integreres der på begge sider med hensyn til $x$

\begin{equation}
    \int \frac{1}{h(y(x))} \cdot \frac{d y(x)}{dx} \,dx = \int g(x)\,dx \,.  \label{foerste}
\end{equation}

Nu betragter vi integranden på venstre side, som hedder $\frac{1}{h(y(x))} \cdot \frac{d y(x)}{dx}$.
Lad os antage at $H(y(x))$ er stamfunktion til $\frac{1}{h(y(x))}$. Da gælder det fra kædereglen at
\begin{equation}
    \begin{split}
    \frac{d H(y(x))}{dx} &= \frac{d H(y(x))}{d y(x)} \cdot \frac{d y(x)}{dx}\\
                         &= \frac{1}{h(y(x))}\cdot \frac{d y(x)}{dx} \label{anden}
    \end{split}
\end{equation}                         
Højre side af ligning \eqref{anden} er da lig integranden i \eqref{foerste}, så venstre side kan indsættes i stedet.

    $$\int \frac{1}{h(y(x))} \cdot \frac{d y(x)}{dx} \,dx = \int \frac{d H(y(x))}{dx} \,dx\,.$$

    Vi ved alle, at integration og differentiation ophæver hinanden således at

    $$\int \frac{ d H(y(x))}{dx} \, dx = H(y(x))\,.$$

    Nu kan ligning \eqref{foerste} skrives som

    $$H(y(x)) = \int \frac{1}{h(y(x))} \, dy(x) = \int g(x) \, dx \,.$$
    
    Alt dette kan også sammenfattes til
\begin{align*}
    \frac{d y(x)}{dx} &= g(x) \cdot h(y(x)) \\
    &\text{kan løses vha.} \\
    \int \frac{1}{h(y)} \, dy &= \int g(x) \, dx
\end{align*}


\end{minipage}
} \fbox{
    \begin{minipage}{0.48\textwidth}
Lad os antage, at vi har en 1. ordens ordinær differentialligning:

$$\frac{d y}{dx} = y^2\cdot 2 x\,.$$

Denne omarrangeres til

$$\frac{1}{y^2} \cdot \frac{d y}{dx} = 2x\,.$$

Nu integreres der på begge sider med hensyn til $x$

\begin{align*}
    \int \frac{1}{y^2} \cdot \frac{d y}{dx} \,dx &= \int 2x\,dx \to \\
    \int \frac{1}{y^2} \ dy &= \int 2x \, dx
\end{align*}
Hver side integreres for sig:
\begin{align*}
\int \frac{1}{y^2} \ dy &= \int 2x \, dx \\
    \frac{y^{-1}}{-1} +k_1 &= \frac{2 x^2}{2} + k_2 \to \\
    \frac{1}{y} &= -x^2 - k_2 + k_1 \to \\
    y = y(x) &=- \frac{1}{x^2 + k_3} \, \text{, hvor }\, k_3= k_2 -k_1
\end{align*}
\vfill
\end{minipage}
}
\end{document}